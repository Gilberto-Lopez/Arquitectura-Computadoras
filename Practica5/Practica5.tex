\documentclass[12pt]{article}
\usepackage{listings}
\usepackage{ucs}
\usepackage[utf8x]{inputenc}
\usepackage[spanish]{babel}
\usepackage{amssymb}
\usepackage[fleqn]{amsmath}
\usepackage{program}

\title{Organización y Arquitectura\\
  de Computadoras\\
  Práctica 5: Arquitectura final.}
\author{Gilberto Isaac López García\\
  Número de cuenta:\\
  Correo: gilisaac@ciencias.unam.mx}
\date{}

\begin{document}

\maketitle

\abstract{La práctica es una arquitectua completa que consta de una ALU, una memoria y una unidad de control diseñadas en \texttt{Verilog} con un programa cargado en memoria que calcula la integral del polinomio $3x^2-2x-2$ en el intervalo $[-2,2]$.}

\section*{Arquitectura}

La arquitectua consta de una ALU con todas las operaciones de la práctica 3, la unidad de control con las instrucciones de la práctica 4 y una memoria de 1024 registros de palabras de 32 bits.

La memoria tiene cargada un programa para aproximar la integral del polinomio $f(x) = 3x^2-2x-2$ en el intervalo $[-2,2]$, mediante sumas de Riemann.

La integral se aproxima como:

$$\int_{-2}^{2} f(x) dx \approx \sum_{n = 0}^{k} f(-2+n\Delta)\Delta$$

donde $\Delta = 0.001$ y $k = 4/0.001 -1$.

El programa es el siguiente:

\begin{program}
  \BEGIN \\ %
    a = 3;
    b = -2;
    c = -2; %
    \rcomment{Los coeficientes del polinomio.}
    d = -2;
    e = 2; %
    \rcomment{Los límites del intervalo de integración.}
    \Delta = 0.001; %
    \rcomment{El tamaño de la partición del intervalo.}
    m = d;
    s = 0;
    \WHILE m < e \DO
    x = a*m^2 + b*m + c;
    s\ += x;
    m\ += \Delta;
  \END
  s\ *= \Delta;
  |return|\ s;
\END
\end{program}

El prgrama en memoria se ve así:

\begin{center}
  \begin{tabular}{ l  l  r }
    Dirección & Instrucción/Dato & Binario \\ \hline
    0 & J 8 & 0x68100000\\
    1 & 3 & 0xC00\\
    2 & -2 & 0x80000800\\
    3 & -2 & 0x80000800\\
    4 & -2 & 0x80000800\\
    5 & 2 & 0x800\\
    6 & 0.001 & 0x1\\
    7 & 0 & 0x0\\
    8 & LW 2 1 & 0x59002000\\
    9 & LW 3 2 & 0x59804000\\
    10 & LW 4 3 & 0x5A006000\\
    11 & LW 5 4 & 0x5A808000\\
    12 & LW 6 5 & 0x5B00A000\\
    13 & LW 7 6 & 0x5B80C000\\
    14 & BLE 5 6 18 & 0x7AB02400\\
    15 & MUL 11 7 11& 0x1DBD8000\\
    16 & SW 11 7 & 0x6580E000\\
    17 & J 26 & 0x68340000\\
    18 & MUL 5 5 8 & 0x1AAC0000\\
    19 & MUL 2 8 8 & 0x19440000\\
    20 & MUL 3 5 9 & 0x19AC8000\\
    21 & SUM 8 9 10 & 0xC4D0000\\
    22 & SUM 10 4 10 & 0xD250000\\
    23 & SUM 11 10 11 & 0xDD58000\\
    24 & SUM 5 7 5 & 0xABA8000\\
    25 & J 14 & 0x681C0000\\
    26 & & \\
  \end{tabular}
\end{center}

\end{document}
